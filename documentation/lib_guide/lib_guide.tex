\documentclass[12pt]{article}

\usepackage{xcolor}
\usepackage{listings}

\lstdefinestyle{customc}{
	belowcaptionskip=1\baselineskip,
	breaklines=true,
	frame=L,
	xleftmargin=\parindent,
	language=C++,
	showstringspaces=false,
	basicstyle=\footnotesize\ttfamily,
	keywordstyle=\bfseries\color{green!40!black},
	commentstyle=\itshape\color{purple!40!black},
	identifierstyle=\color{blue},
	stringstyle=\color{orange},
}

\lstdefinestyle{customasm}{
	belowcaptionskip=1\baselineskip,
	frame=L,
	xleftmargin=\parindent,
	language=[x86masm]Assembler,
	basicstyle=\footnotesize\ttfamily,
	commentstyle=\itshape\color{purple!40!black},
}

\lstset{escapechar=@,style=customc}
%opening
\title{APRLib Documentation}
\author
{Bevan L. Cheeseman}

\begin{document}

\maketitle

\section{Installation}
see Readme.md

\section{Calculating the APR}
For any dataset, the first step is transforming the original image into an Adaptive Particle Representation. This is achieved using the APRConverter class, and can be achieved using the Example\_get\_apr example, found in test/Examples/.

The example takes a unsigned 16 bit, unsigned 8 bit, or float precision tiff image, and produces the APR, output to  *\_apr.h5 file (readable by all other examples and HDFView, and matlab reader (see Matlab\_scripts)) and a down sampled image of the Particle Cell level of the computed APR. This is useful for assessing what content was captured the APR, and can be used for optimizing parameters.
\subsection{Example\_get\_apr usage}
For basic using and description run the executable with no options.
\subsection{Parameter Selection}
By default the required parameters will be automatically estimated from the input image. This is done by estimating the background noise level, and assuming the dataset has a minimum SNR of six. 

The required parameters can be grouped into two. Those that can be used to focus the APR and remove un-wanted background content (flouresence), namely the intensity threshold (-I\_th), and minimum signal level (-min\_signal), and those that impact the local adaptation of the APR, namely the requried relative error $E$ (-rel\_error) and smoothing parameter $\lambda$ (-lambda). We recommend that the user focuses on adaptation of the first two parameters, before optimizing the second group. Lastly, the APR can be additionally guided by the use of a binary input mask (-mask\_file).

Below we give examples of the behaviour of the automatic parameters and guidance on how to interpret and set parameters for your given dataset.
\subsection{Examples}
Auto parameters.
\subsection{Using I\_th and and min\_signal to guide the APR}

\subsection{Low noise, or processed datasets}

\subsection{Guiding the APR with a mask file}

\section{Processing with the APR}

\subsection{Data structure}

\subsection{Input-Output}

\section{Data Access and Processing}


\subsection{Serial Iteration}

\begin{lstlisting}
for (apr.begin(); apr.end() != 0; apr.it_forward()) {
float particle_intensity = apr(apr.particles_intensities);
}
\end{lstlisting}

\subsection{Parallel Iteration}


\subsection{Neighbor Iteration}


\subsection{Particle Transforms}

\section{Pixel Images and Reconstruction}
\subsection{Mesh\_data class}

\subsection{Input - Output}

\subsection{Reconstruction}

\section{MISC}











\end{document}
